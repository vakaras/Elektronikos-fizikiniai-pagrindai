\subsection{Slinkties srovė}

Kintama srovė tekanti tarp kondensatoriaus plokštelių yra vadinama
slinkties srove:
\begin{align*}
  j &= \varepsilon \varepsilon_{0} \frac{\delta E}{\delta t} \\
\end{align*}

\subsection{Galia}

\begin{align*}
  S
  &= \sqrt{\left( U_{a}I \right)^{2} +
    \left( U_{r}I \right)^{\frac{1}{2}}}
\end{align*}
čia:
\begin{description}
  \item[$S$] – pilnoji galia;
  \item[$U_{r}$] – reaktyvinė įtampa;
  \item[$U_{a}$] – aktyvinė įtampa.
\end{description}


% INCLUDE: 

Kadangi visi sujungti nuosekliai, tai srovės stipris visose dalyse
yra lygus (pažymėkime $I$). Kadangi kondensatoriuje įtampa atsilieka
nuo srovės stiprio, per $\frac{\pi}{2}$, atidedame $U_{C}$. Ritėje
įtampa aplenkia srovės stiprį per $\frac{\pi}{2}$, atidedame $U_{L}$.
Paprastoje varžoje fazių poslinkio nėra, pažymime $U_{R}$. Atstojamoji
grandinės $U$ bus lygi vektorinei $U_{C}$, $U_{L}$ ir $U_{R}$ sumai.

Srovės laidumas:
\begin{align*}
  I
  &= \frac{U}{Z} \\
  &= G \cdot U \\
\end{align*}

\subsection{Kompleksiniai skaičiai}

Kompleksinių skaičių žymėjimai:
\.{a}, $a$ (su brūkšniuku viršuje, arba apačioje), \textbf{a}. FIXME

\begin{align*}
  a
  &= a \cos \varphi + ja \sin \varphi \\
  &= a (\cos \varphi + j \sin \varphi) \\
\end{align*}

Algebrinė kompleksinio skaičiaus forma:
\begin{align*}
  a &= a' + j a'' \\
\end{align*}
čia:
\begin{description}
  \item[$a'$] – projekcija realiojoje ašyje ($a' = a \cos \varphi$);
  \item[$a''$] – projekcija menamojoje ašyje ($a'' = a \sin \varphi$).
\end{description}

Kompleksinių skaičių daugyba:
\begin{align*}
  a \cdot b
  &= a e^{j\varphi} \cdot b e^{j\alpha} \\
  \begin{cases}
    \sin \varphi = \frac{e^{j\varphi} - e^{-j\varphi}}{2j} \\
    \cos \varphi = \frac{e^{j\varphi} + e^{-j\varphi}}{2} \\
  \end{cases} \\
\end{align*}

Pavyzdys su nuoseklia grandinė, kur srovės stipris yra $I_{0}$:
\begin{equation*}
  I_{0} R + I_{0} j \omega L + I_{0} \frac{1}{j \omega C} =
  I_{0} R + I_{0} j \left( \omega L - \frac{1}{\omega C} \right)
\end{equation*}
