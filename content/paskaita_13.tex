\subsection{pn riba}

Suglauskime elektroninį ir skylinį puslaidininkius. Elektronai iš n
pereis į p tipo, ten kur jų koncentracija yra gerokai mažesnė.

Dėl jonų atsiradęs elektrinis laukas trukdo tolesniam elektronų ir
skylių migravimui. Susidaro dinaminė pusiausvyra. Susidaro potencinis
sluoksnis – potencinis barjeras, kurio varža yra didelė. Jo plotis
nuo $10^{-8}$m iki $10^{-5}$m.

Prie p tipo prijunkime baterijos $-$. Išoriniu lauku barjerą padidiname
dar.

Prie p tipo prijunkime baterijos $+$. Barjeras sumažėja, o dar padidinus
šaltinio galingumą, srovė pradeda tekėti laisvai.

TODO: Stabilitronai.

Dvipolis tranzistorius.
\begin{description}
  \item[Emiteris] – priemaišų koncentracija yra didžiausia, didžiausio
    laidumo;
  \item[Bazė] – priemaišų koncentracija pati mažiausia;
  \item[Kolektorius] – priemaišų koncentracija mažesnė nei emiteryje, bet
    didesnė nei kolektoriuje.
\end{description}

Skylių koncentracija bazėje yra nedidelė.

Jei Emiterio srovę pažymėsime:
\begin{align*}
  I_{E} &= I_{C} + I_{B} \\
  \intertext{Iš fakto, kad $I_{B} << I_{C}$}
  I_{E} &\approx I_{C} \\
\end{align*}

Tranzistorius yra elektriškai valdoma varža.

Srovės perdavimo koeficientas:
\begin{align*}
  \beta &= \frac{\Delta I_{C}}{\Delta I_{B}} \\
\end{align*}
Paprastai $\beta$ nuo 20 iki 200. (Mikroschemose iki 5000.)

Įėjimo charakteristika:
\begin{align*}
  I_{B} &= f(U_{BE}) \\
\end{align*}

Išėjimo charakteristika:
\begin{align*}
  I_{C} &= f(U_{CE}) \\
  I_{B} &= const \\
\end{align*}

Laukinis tranzistorius. TODO:
\begin{description}
  \item[D] – santaka;
  \item[S] – ištaka;
  \item[G] – užtara.
\end{description}
