\section{$p-n$ riba}

Suglauskime elektroninį ir skylinį puslaidininkius. Elektronai iš $n$
pereis į $p$ tipo, ten kur jų koncentracija yra gerokai mažesnė. Jie
difunduodami palieka teigiamai įelektrintus jonus (donorinės priemaišos).
Analogiškai iš $p$ į $n$ difunduoja skylės, taip palikdamos neigiamai
įelektrintus jonus. Dėl jonų atsiranda elektrinis laukas ir trukdo
tolesniam elektronų ir skylių migravimui. Taip pat riboje susidaro
dvigubas elektrinis sluoksnis, kuriame nėra nei laisvų elektronų, nei
laisvų skylių. Jo varža didelė, storis gali būti nuo $10^{-8}m$
iki $10^{-5}m$ – jis vadinamas užtveriamuoju. $p-n$ ribos elektrinis
laukas trukdo pereiti pagrindiniams krūvininkams (elektronams iš
$n$ į $p$ ir skylėms iš $p$ į $n$) ir padeda pereiti nepagrindiniams
(elektronams iš $p$ į $n$ ir skylėms iš $n$ į $p$). Susidaro dinaminė
pusiausvyra.

Prie $p$ prijunkime baterijos $-$, o prie $n$ – $+$. Išorinis laukas bus
tokios pačios krypties, kaip ir vidinis. Užtveriamojo sluoksnio storis
dar padidės. Išorinis laukas pagrindinius krūvininkus ištrauks, o
ne pagrindinių yra mažai ir srovė tekės maža.

Dabar prie $p$ tipo prijunkime baterijos $+$, o prie $n$ – $-$. Riba
yra papildoma krūvininkais, ji siaurėja, varža mažėja, pagrindiniai
krūvininkai laisvai praeina ribą – srovė teka laisvai.

\subsection{$p-n$ sandūros diodai}

\begin{defn}[Diodas]
  Diodą sudaro puslaidininkyje suformuota $p-n$ sandūra ir elektrodų
  išvadai nuo $p$ ir $n$ sričių.
\end{defn}

TODO: Papildyti iš dėstytojo konspekto.

TODO: Voltamperinės charakteristikos brėžinys su paaiškinimais.

TODO: Stabilitronai.

\subsection{Dvipolis tranzistorius}

\begin{defn}[Dvipolis tranzistorius]
  Skirtingo laidumo sričių struktūra, turinti dvi $p-n$ sandūras.
\end{defn}

\begin{description}
  \item[Emiteris] – priemaišų koncentracija yra didžiausia, didžiausio
    laidumo;
  \item[Bazė] – laidumo tipas skiriasi, nuo kraštinių sričių, o priemaišų
    koncentracija yra pati mažiausia;
  \item[Kolektorius] – priemaišų koncentracija mažesnė nei emiteryje, bet
    didesnė nei kolektoriuje.
\end{description}

Bazės storis yra daug mažesnis už krūvininkų difuzijos nuotolį:
$\delta_{B} = 100 \mu m$ pavieniame tranzistoriuje ir $~0,1 \mu m$
integrinės mikroschemos tranzistoriuje.

Abiejų $p-n$ sandūrų tiesioginės kryptys yra priešingos: $n-p-n$
tipo tranzistoriaus yra $B \to E$ ir $B \to C$, o $p-n-p$ –
$E \to B$ ir $C \to B$. Sutartiniuose ženkluose emiterio rodyklė yra
nukreipta $B-E$ sandūros tiesiogine kryptimi (iš $p$ į $n$).

TODO: Brėžinys iš dėstytojo konspektų.
Dėl plonos bazės tik $1-5\%$ elektronų bazėje rekombinuoja, o likusius
surenka kolektorius. Skyles (bazėje) rekombinuojančias su elektronais
atstato baterija. Kol $U_{BE} = 0$, sandūroje $BE$ srovė neteka. Kadangi
$U_{CE}$ yra atgalinė $BC$ sandūros įtampa, tai ir kolektoriaus
grandinėje srovė neteka.

Paduokime į bazę teigiamą potencialą, ši įtampa $U_{BE}$ yra tiesioginė
sandūros $BE$ įtampa, todėl iš $E$ į $B$ pradeda tekėti elektronai
(pagrindiniai) ir didelė dalis yra surenkama kolektoriaus, kadangi
$U_{CB} > 0$ ir $U_{CB} >> U_{BE}$. Jei Emiterio srovę pažymėsime $I_{E}$:
\begin{align*}
  I_{E} &= I_{C} + I_{B} \\
  \intertext{iš fakto, kad $I_{B} << I_{C}$}
  I_{E} &\approx I_{C}. \\
\end{align*}
Taigi, bazės grandinės srove valdome kolektoriaus srovę. Priklausomybė
$I_{C} = f(I_{B})$ yra vadinama \emph{perdavimo charakteristika}, o
$\beta = \frac{\Delta I_{C}}{\Delta I_{B}}$ – srovės perdavimo
koeficientu. Tranzistorių $\beta$ gali būti nuo 20 iki 200, o
mikroschemose net iki 5000.

Kai $U_{CE} = const$, įėjimo charakteristika:
\begin{align*}
  I_{B} &= f(U_{BE}) \\
\end{align*}
TODO: Brėžinys iš dėstytojo konspektų.

Kai $I_{B} = const$, išėjimo charakteristika:
\begin{align*}
  I_{C} &= f(U_{CE}) \\
\end{align*}
TODO: Brėžinys iš dėstytojo konspektų.

Iš brėžinių matome, kad kolektoriaus soties srovė nepriklauso nuo
įtampos $U_{CE}$, o priklauso tik nuo $I_{B}$.

Kiekvieną tranzistorių galima laikyti valdomu netiesiniu rezistoriumi,
kurio voltamperinės charakteristikos (išėjimo) yra keičiamos bazės
srove. Kitaip tariant, tranzistorius yra elektriškai valdoma varža.

\subsection{Lauko tranzistorius}

Lauko tranzistorius yra tranzistorius, kurio srovė yra valdoma
potencialu.

TODO: Brėžiniai iš dėstytojo konspektų.

TODO:
\begin{description}
  \item[D] – santaka;
  \item[S] – ištaka;
  \item[G] – užtūra.
\end{description}
