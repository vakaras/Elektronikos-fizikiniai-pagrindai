\begin{defn}[Kampinis dažnis]
  % TODO Išsiaiškinti kuo skiriasi nuo kampinio greičio.
  \begin{align*}
    w
    &=& 2 \pi \nu \\
    &=& 2 \pi \frac{1}{T} 
  \end{align*}
  čia:
  \begin{description}
    \item[$\nu$] – dažnis;
    \item[$T$] – periodas.
  \end{description}
\end{defn}

\subsection{Saviindukcija}

\begin{align}
  \Phi &= L \cdot I \label{eq:magnetinis_srautas_nuo_stiprio} \\
\end{align}
čia:
\begin{description}
  \item[$\Phi$] – rite tekančios srovės kuriamas magnetinis srautas;
  \item[$I$] – ritėje tekančios srovės stipris;
  \item[$L$] – ritės induktyvumas, kuris priklauso nuo aplinkos savybių
    ($L = 1 \frac{Wb}{A} = 1H$).
\end{description}

Jei per laiko tarpą $dt$, srovės stipris pakito $dI$, tai
iš \ref{eq:magnetinis_srautas_nuo_stiprio} gauname, kad:
\begin{align*}
  \frac{d\Phi}{dt}
    &= L \frac{dI}{dt} \\
  \intertext{pagal Faradėjaus dėsnį žinome, kad:}
  E_{i}
    &= - \frac{d\Phi}{dt} \\
  \intertext{kadangi ši srovė $E_{i}$ yra indukuota srovės kitimo,
  tai ją vadinsime saviindukcine srove ir žymėsime $E_{s}$:}
  E_{s}
    &= E_{i} \\
    &= - L \frac{dI}{dt} \\
  \intertext{iš Omo dėsnio žinome, kad srovės stipris yra:}
  I_{i}
    &= \frac{E_{i}}{R} \\
    &= - \underbrace{\frac{1}{R}}_{\sigma} \frac{d\Phi}{dt} \\
  \intertext{taigi indukuotas srovės stipris yra:}
  I_{s}
    &=  \frac{E_{s}}{R} \\
    &= - \frac{L}{R} \frac{dI}{dt} \\
  \intertext{bendras grandinėje tekantis srovės stipris yra:}
  I
    &= \frac{E + E_{s}}{R} \\
    &= \frac{1}{R} \left( E - L \frac{dI}{dt} \right) \\
  \intertext{taigi gavome, kad saviindukcijos srovė priešinasi ją
  sukėlusios srovės kitimui.}
\end{align*}
čia:
\begin{description}
  \item[$\sigma$] – savitasis laidumas;
\end{description}

\subsection{Sukūrinės Fuko srovės}

Jei laidininkas nėra plonas, pavyzdžiui kokio nors metalo gabalas, tai jame
gali susidaryti sukūrinės srovės. Jų stiprumas:
\begin{align*}
  I_{s} &= \frac{E_{s}}{R} \\
  &= - \frac{1}{R} \frac{d \Phi}{dt}
\end{align*}
FIXME: Prie ko šita formulė?

\emph{Paviršinis efektas (skinefektas):} Didelio dažnio srovė teka tik
laidininko paviršiumi. TODO: Paaiškinti kodėl.

Dėl šių priežasčių, vietoj to kad būtų gaminami stori vientisi laidininkai,
gaminami „supinti“ iš plonų laidelių.

\subsection{Srovės stiprio priklausomybė nuo laiko}

Anksčiau gavome, kad srovės stiprio kitimas išreiškiamas formule:
\begin{align*}
  I
    &= \frac{1}{R} \left( E - L \frac{d I}{d t} \right) \\
  \intertext{ją pertvarkome:}
  I \cdot R
    &= E - L \frac{dI}{dt} \\
  L \frac{dI}{dt}
    &= E - IR \\
  \frac{d I}{E - I R} &= \frac{1}{L} dt \\
  \intertext{tarę, kad $E, R, L$ yra konstantos, integruojame:}
  - \frac{\ln\left( E - IR \right)}{R}
    &= \frac{t}{L} - \ln C \\
  \ln \left( E - IR \right)
    &= - \frac{R}{L}t + \ln C \\
    &= \ln e^{-\frac{R}{L}t} + \ln C \\
    &= \ln \left( C \cdot e^{-\frac{R}{L}t} \right) \\
  E - IR
    &= C \cdot e^{-\frac{R}{L}t}
  \intertext{kai $t = 0$, tada $I = I_{0}$ ir iš to seka, kad
  $C = E - I_{0}R$.  Įrašę atgal į formulę, gauname:}
  E - IR
    &= \left( E - I_{0}R \right) e^{-\frac{R}{L}t} \\
    &= E e^{-\frac{R}{L}t} - I_{0}Re^{-\frac{R}{L}t} \\
  IR
    &= I_{0}Re^{-\frac{R}{L}t} + E - E e^{-\frac{R}{L}t} \\
    &= I_{0}Re^{-\frac{R}{L}t} + E \left( 1 - e^{-\frac{R}{L}t} \right) \\
  I &= I_{0}e^{-\frac{R}{L}t} + \frac{E}{R}
    \left( 1 - e^{-\frac{R}{L}t}\right) \\
  \intertext{Taigi įjungimo momentu elektros srovės stiprio priklausomybė
  nuo laiko apibūdinama formule:}
  I &= \frac{E}{R} \left( 1 - e^{-\frac{R}{L}t} \right) \\
\end{align*}

\subsection{Savitarpio indukcija}

Tarkime, jog turime dvi rites užmautas ant to paties žiedo.
(TODO: Brėžinys.) Jei per pirmąją teka srovė, kurios stipris kinta
$\frac{dI_{1}}{dt}$, tai žiede atsiranda magnetinis srautas, kurio
stiprumas yra:
\begin{align*}
  \frac{d \Phi}{dt}
    &= L_{1}\frac{dI_{1}}{dt} \\
  \intertext{jis antrojoje ritėje indukuoja elektrovarą:}
  E_{2}
    &= - \frac{d \Phi}{dt} \\
    &= - L_{1}\frac{dI_{1}}{dt} \\
  \intertext{Žinome, kad jei ritėje yra $N$ vijų, tai elektrovara}
  E
    &= - N \frac{d \Phi}{dt} \\
  \intertext{Padarę prielaidą, kad energijos praradimų nėra, iš
  simetriškumo gauname, kad:}
  E_{1}
    &= - N_{1} \frac{d\Phi}{dt} \\
  E_{2}
    &= - N_{2} \frac{d\Phi}{dt} \\
  \frac{E_{1}}{E_{2}}
    &= \frac{N_{1}}{N_{2}} \\
    &= k \\
  \intertext{gautasis $k$ vadinamas transformatoriaus transformavimo 
  koeficientu. Kadangi laidininko išspinduliuojamas šilumos kiekis:}
  Q &=
    I^{2} R t
  \intertext{tai vietoj to, kad gaminti storesnius laidus, labiau
  apsimoka prieš perduodant didelius atstumus aukštinamaisiais
  transformatoriais paaukštinti įtampą (taip sumažinant srovės stiprį).
  Srovės galingumas:}
  P &=
    E I
\end{align*}

\subsection{Elektrinių ir magnetinių laukų energija}

Darbas elektriniame lauke perkeliant krūvį $q$ yra lygus:
\begin{align*}
  A &=
    q \cdot U \\
  \intertext{kondensatoriaus krūvis, kai jo talpa yra $C$ ir įtampa
  tarp plokštelių yra $U$:}
  Q &=
    C \cdot U \\
  \intertext{kondensatoriaus elektrinio lauko energija:}
  W_{e}
    &= \int _{0} ^{Q} U \cdot dq \\
    &= \frac{1}{2} \frac{Q^2}{C} \\
    &= \frac{1}{2} C U^{2} \\
\end{align*}

Atliktas darbas srovei stiprėjant nuo 0 iki $I$:
\begin{align*}
  A
    &= \int P dt \\
  \intertext{čia $P$ yra galia, kuri yra lygi:}
  P
    &= E I \\
    &= I\left( - \frac{d\Phi}{dt} \right) \\
    &= I\left( -L \frac{dI}{dt} \right) \\
  \intertext{$-$ tiesiog rodo priešingą kryptį, todėl skaičiuodami galią
  galime ją ignoruoti:}
  P
    &= IL \frac{dI}{dt} \\
  \intertext{Magnetinio lauko energija $W_{m}$ yra lygi visam atliktam
  darbui:}
  W_{m}
    &= A \\
    &= \int P dt \\
    &= \int_{0}^{I} LI dI \\
    &= \frac{LI^{2}}{2} \\
\end{align*}

\section{Kintamoji srovė}

\subsection{Omo dėsnis kintamajai srovei}

\ref{fig:grandine:RLC} paveikslėlyje pateiktoje grandinėje nuolatinė
elektros srovė teka tik tol, kol pasikrauna kondensatorius $C$. Tuo
tarpu, kintamoji srovė šioje grandinėje teka.

\begin{figure}[H]
  \begin{center}
    \begin{circuitikz}[scale=1.2]\draw
    (0,2) -- (0, 4)
          to[generic, l=$R$] (2, 4)
          to[L, l=$L$] (4, 4)
          to[C, l=$C$] (6, 4)
          to[ammeter] (6, 2)
          -- (4, 2)
          to[vsourcesin, l=$U$] (2, 2)
          -- (0, 2)
    (2,2) -- (2, 0)
          to[voltmeter] (4, 0)
          -- (4, 2)
    ;
    \end{circuitikz}
  \end{center}
  \caption{Grandinė.}
  \label{fig:grandine:RLC}
\end{figure}

Dabar grandinėje palikime tik $R$ ir $L$ (\ref{fig:grandine:RL}
paveikslėlis). Iš pradžių prijunkime nuolatinę srovę, kurios įtampa yra
$E$, ir išmatuokime srovės stiprį $I_{1}$. Tada prijunkime kintamąją
srovę, kurios įtampa $E_{\t{efektinė}} (E_{\t{efektinė}} = E)$, ir
išmatuokime $I_{2}$. Gausime, kad $I_{2} < I_{1}$. Taigi ritė sudaro
varžą, kuri vadinama induktyviąja ir žymima $X_{L}$:
\begin{equation}
  X_{L} = \omega L
  \label{eq:induktyvioji_varza}
\end{equation}
čia:
\begin{description}
  \item[$\omega$] – kampinis dažnis, $\omega = \frac{2\pi}{T}$,
    kur $T$ – periodas.
\end{description}

\begin{figure}[H]
  \begin{center}
    \begin{circuitikz}[scale=1.2]\draw
    (0,2) -- (0, 4)
          to[generic, l=$R$] (3, 4)
          to[L, l=$L$] (6, 4)
          to[ammeter] (6, 2)
          -- (4, 2)
          to[vsourcesin, l=$U$] (2, 2)
          -- (0, 2)
    (2,2) -- (2, 0)
          to[voltmeter] (4, 0)
          -- (4, 2)
    ;
    \end{circuitikz}
  \end{center}
  \caption{Grandinė.}
  \label{fig:grandine:RL}
\end{figure}

Dabar palikime grandinėje tik $R$ ir $C$ (\ref{fig:grandine:RC}
paveikslėlis). Nuolatinė srovė šioje grandinėje neteka, o kintamai
srovei tarp kondensatoriaus plokščių susidaro kintamas elektrinis
laukas sudarantis kondensatoriaus dielektrike \emph{slinkties srovę},
kuri yra laidumo srovės tęsinys. Slinkties srovės tankis yra
\begin{equation*}
  j_{sl} = \frac{d \sigma}{dt}
\end{equation*}
čia:
\begin{description}
  \item[$\sigma$] – elektrodo paviršinio krūvio tankis.
\end{description}
\begin{figure}[H]
  \begin{center}
    \begin{circuitikz}[scale=1.2]\draw
    (0,2) -- (0, 4)
          to[generic, l=$R$] (3, 4)
          to[C, l=$C$] (6, 4)
          to[ammeter] (6, 2)
          -- (4, 2)
          to[vsourcesin, l=$U$] (2, 2)
          -- (0, 2)
    (2,2) -- (2, 0)
          to[voltmeter] (4, 0)
          -- (4, 2)
    ;
    \end{circuitikz}
  \end{center}
  \caption{Grandinė.}
  \label{fig:grandine:RC}
\end{figure}
Varža kurią sudaro kondensatorius kintamai srovei vadinama talpine:
\begin{equation*}
  X_{C} = \frac{1}{\omega C},
\end{equation*}
čia:
\begin{description}
  \item[$\omega$] – kampinis dažnis;
  \item[$C$] – kondensatoriaus talpa.
\end{description}

Dabar vėl sujunkime \ref{fig:grandine:RLC} paveikslėlyje pateiktą grandinę
ir raskime minėtus dėsningumus matematiškai. Kai į grandinę įjungta tik
$R$, tai fazių skirtumas tarp srovės ir įtampos yra lygus 0. Tai yra:
\begin{align*}
  \intertext{jei}
  E
    &= E_{0} \sin \omega t \\
  \intertext{tai}
  I
    &= \frac{E}{R} \\
    &= \frac{E_{0}}{R} \sin \omega t \\
  \intertext{tuo tarpu kondensatoriuje, kadangi}
  q
    &= CE \\
    &= C E_{0} \sin \omega t \\
  \intertext{ir srovės stipris yra krūvio išvestinė}
  I_{C}
    &= C E_{0} \omega \cos \omega t \\
    &= C E_{0} \omega \sin \left( \omega t + \frac{\pi}{2} \right) \\
  \intertext{gavome, kad srovės stipris kondensatoriuje aplenkia įtampa
  $\frac{\pi}{2}$. Amplitudžių $E_{0}$ ir $I_{0}$ santykis yra lygus
  varžai:}
  X_{C}
    &= \frac{E_{0}}{I_{0}} \\
    &= \frac{E_{0}}{C \omega E_{0}} \\
    &= \frac{1}{\omega C} \\
\end{align*}
Dabar tarkime, kad per ritę tekantis srovės stipris yra:
\begin{align*}
  I
    &= I_{0} \sin \omega t \\
  \intertext{žinome, kad kintamoji įtampa $U$ ir saviindukcinė
  elektrovaros jėga $E_{I}$ yra susijusios taip:}
  U
    &= E_{I} \\
  \intertext{bet}
  E_{I}
    &= - \frac{d \Phi}{d t} \\
    &= - L \frac{d I}{d t} \\
    &= - L I_{0} \omega \cos \omega t \\
    &= - L I_{0} \omega \sin \left( \omega t + \frac{\pi}{2} \right) \\
  \intertext{taigi}
  U
    &= L I_{0} \omega \sin \left( \omega t + \frac{\pi}{2} \right) \\
  \intertext{gavome, kad ritėje srovės stipris atsilieka nuo įtampos
  per $\frac{\pi}{2}$.}
\end{align*}

Pilnos įtampos momentinės vertės:
\begin{align*}
  E
    &= E_{R} + E_{C} + E_{L} \\
    &=  I_{0} R \sin \omega t +
        I_{0} X_{C} \sin \left( \omega t - \frac{\pi}{2} \right) +
        I_{0} X_{L} \sin \left( \omega t + \frac{\pi}{2} \right) \\
    &=  I_{0}R \sin wt + I_{0} X_{L} \cos wt - I_{0} X_{C} \cos wt \\
  \frac{E}{I_{0}}
    &= R \sin wt + (X_{L} - X_{C})\cos wt \\
  \intertext{daliname abi lygybės puses iš impedanco (atstojamosios
  grandinės varžos, žr. brėžinį):}
  \frac{E}{I_{0}Z}
    &= \frac{R}{Z} \sin wt + \frac{X_{L} + X_{C}}{Z} \cos wt \\
  \intertext{iš brėžinio matome, kad}
  \frac{R}{Z}
    &= \cos \varphi \\
  \frac{X_{L} - X_{C}}{Z}
    &= \sin \varphi \\
  \intertext{pakeitę, gauname:}
  \frac{E}{I_{0}Z}
    &= \cos \varphi \sin \omega t + \sin \varphi \cos \omega t \\
  E
    &= I_{0} Z ( \cos \varphi \sin wt + \sin \varphi \cos wt ) \\
  \intertext{čia:}
  Z
    &= \sqrt{R^{2} + \left( X_{L} - X_{C} \right)^{2}} \\
\end{align*}

\begin{verbatim}
   /|
Z / |
 /  | (X_{L} - X_{C})
/___|
  R
\end{verbatim}

$Z$ įgyja minimumą, kai
\begin{align*}
  X_{L} - X_{C}
    &= 0 \\
  \omega_{0} L - \frac{1}{\omega_{0}C}
    &= 0 \\
  \omega_{0}
    &= \frac{1}{\sqrt{LC}} \\
    &= 2 \pi \frac{1}{T} \\
\end{align*}
čia:
\begin{description}
  \item[$\omega_{0}$] – rezonansinis kampinis dažnis;
  \item[$T$] – periodas.
\end{description}

Iš čia gauname Tomsono formulę:
\begin{equation*}
  T = 2 \pi \sqrt{LC}
\end{equation*}

\subsection{Kintamos srovės galia}

Momentinė galia:
\begin{align*}
  p &= e \cdot i \\
  \intertext{sinusinei srovei:}
  i &= I_{0} \sin \omega t \\
  e &= E_{0} \sin \left( \omega t + \varphi \right) \\
  \intertext{taigi}
  p &= E_{0} I_{0} \sin \left( \omega t + \varphi \right) \sin \omega t \\
\end{align*}

TODO: dėstytojo konspektų 2 psl.

\section{Sinusinių dydžių vaizdavimas vektoriais}

\begin{align*}
  \sin (wt + \varphi) &=& \frac{i}{I_{m}} \\
\end{align*}
