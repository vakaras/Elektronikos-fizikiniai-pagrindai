\begin{defn}[Kampinis dažnis]
  % TODO Išsiaiškinti kuo skiriasi nuo kampinio greičio.
  \begin{align*}
    w
    &=& 2 \pi \nu \\
    &=& 2 \pi \frac{1}{T} 
  \end{align*}
  čia:
  \begin{description}
    \item[$\nu$] – dažnis;
    \item[$T$] – periodas.
  \end{description}
\end{defn}

\section{Saviindukcija}

\begin{align*}
  \Phi &=& L \cdot I
\end{align*}
čia:
\begin{description}
  \item[$L$] – ? kuris priklauso nuo aplinkos savybių.
    $L = 1 \frac{Wb}{A} = 1H$
\end{description}

\begin{align*}
  \frac{d\Phi}{dt} &=& L \frac{dI}{dt} \\
  E_{i} &=& - \frac{d\Phi}{dt} \\
  E_{i} &=& E_{s} \\
  &=& - L \frac{dI}{dt} \\
  I &=& \frac{E_{i}}{R} \\
  &=& - \underbrace{\frac{1}{R}}_{\sigma} \frac{d\Phi}{dt}
  I_{s} &=&  \frac{E_{s}}{R} \\
  &=& - \frac{L}{R} \frac{dI}{dt}
  I &=& \frac{E + E_{s}}{R} \\
  &=& \frac{1}{R} \left( E - L \frac{dI}{dt} \right) \\
\end{align*}
čia:
\begin{description}
  \item[$\sigma$] – savitasis laidumas;
\end{description}

\section{Sukūrinės Fuko srovės}

Jei laidininkas nėra plonas, pavyzdžiui kokio nors metalo gabalas, tai jame
gali susidaryti sukūrinės srovės.
\begin{align*}
  I_{s} &=& \frac{E_{s}}{R} \\
  &=& - \frac{1}{R} \frac{d \Phi}{dt}
\end{align*}

\section{Srovės stiprio priklausomybė nuo laiko}

\begin{align*}
  \frac{d I}{E - I R} &=& \frac{1}{L} dt
\end{align*}
Tarę, kad $E, R, L$ yra konstantos, integruojame:
\begin{align*}
  \ln \left( E - IR \right) &=& - \frac{R}{L}t + \ln C
\end{align*}
Potencijuojame:
\begin{align*}
  E - IR &=& C \cdot e^{-\frac{R}{L}t}
\end{align*}

Kai $t = 0$, tada $I = I_{0}$. Tada $C = E - IR$. Gauname:
\begin{align*}
  I &=& I_{0}e^{-\frac{R}{L}t} + \frac{E}{R}
    \left( 1 - e^{-\frac{R}{L}t}\right) \\
  I &=& \frac{E}{R} \left( 1 - e^{-\frac{R}{L}t} \right)
    & \t{Įjungimo momentu.}
\end{align*}

\section{Savitarpio indukcija}

\begin{align*}
  E_{2} &=& -\frac{d\Phi_{21}}{dt} \\
  &=& - L_{21} \frac{dI_{1}}{dt}
\end{align*}

% INCLUDE: Brėžinį iš Audriaus sąsiuvinio. ID=#0005

\begin{align*}
  e_{1} &=& - N_{1} \frac{d \Phi}{dt} \\
  e_{2} &=& - N_{2} \frac{d \Phi}{dt}
\end{align*}

Transformatoriaus transformavimo koeficientas:
\begin{align*}
  \frac{e_{1}}{e_{2}} = \frac{U_{1}}{U_{2}} = \frac{N_{1}}{N_{2}} = k
\end{align*}

Laidininko išspinduliuojamas šilumos kiekis:
\begin{align*}
  Q &=& I^{2} R t
\end{align*}

Srovės galingumas:
\begin{align*}
  P &=& E I
\end{align*}

\section{Elektrinių ir magnetinių laukų energija}

Darbas elektriniame lauke perkeliant krūvį:
\begin{align*}
  A &=& q \cdot U \\
  q &=& C \cdot U \\
  E_{e} &=& \int _{0} ^{Q} U \cdot dq \\
  &=& \frac{1}{2} \frac{Q^2}{C} \\
  E_{e} &=& \frac{1}{2} C U^{2}
\end{align*}

Atliktas darbas srovei stiprėjant nuo 0 iki $I$:
\begin{align*}
  P &=& I L \frac{dI}{dt} \\
  \int d E_{m} &=& \int d A \\
  &=& \int P dt \\
  &=& \int _{0} ^{I} L I dI \\
  E_{m} &=& \frac{1}{2} L I^{2}
\end{align*}

\chapter{Kintamoji srovė}

\section{Omo dėsnis kintamajai srovei}

Induktyvioji (reaktyvioji) varža:
\begin{align*}
  X_{L} &=& w L
\end{align*}
čia:
\begin{description}
  \item[$w$] – kampinis dažnis.
\end{description}

\begin{defn}[Slinkties srovė]
  TODO
\end{defn}

\section{Sinusinių dydžių vaizdavimas vektoriais}

\begin{align*}
  \sin (wt + \varphi) &=& \frac{i}{I_{m}} \\
\end{align*}
