\chapter{Elektrostatika}

\begin{description}
  \item[$\vec{E}$] – elektrinio lauko stipris;
  \item[$\vec{D}$] – elektrostatinė indukcija.
\end{description}

\begin{align*}
  \vec{E} &= \frac{\vec{F}}{q} \\
  \vec{D} &= \varepsilon_{0} \varepsilon \vec{E}
\end{align*}

\begin{defn}[srautas]
  \begin{equation}
    \Phi = \vec{E} \cdot \vec{S} \cdot \cos \alpha
    \label{def:srautas}
  \end{equation}
  \begin{description}
    \item[$\vec{S}$] – plotas (pseudo vektorius, kryptis sutampa su
      normalės kryptimi);
    \item[$\vec{E}$] – elektrinio lauko stipris.
  \end{description}
\end{defn}

\begin{description}
  \item[$A$] – darbas;
  \item[$r$] – kelias.
\end{description}

\begin{align*}
  \Psi &= \vec{D} \vec{S} = \sum _{i=1} ^{n} q_{i} \\
  \Phi &= \oint_{S} \vec{E} d \vec{S} = 
    \frac{1}{\varepsilon_{0}} \int _{v} \rho dv \\
  \delta A &= \vec{F} \cdot \vec{d} \cdot r =
    q \cdot \vec{E} \vec{d} r = -d W p
\end{align*}

Elektrinio lauko potencialias:
\begin{align*}
  \varphi &= \frac{Wp}{q} & \text{vienam taškui} \\
  \varphi &= \sum ^{n} _{i=1} \frac{q_{i}}{4 \pi \varepsilon_{0} r} 
    & \text{taškų sistemai}.
\end{align*}

Darbas perkeliant krūvį iš taško su potencialu $\varphi_{1}$ į tašką
su potencialu $\varphi_{2}$ lygus:
\begin{align*}
  A_{1-2} &= q \cdot (\varphi_{1} - \varphi_{2}) \\
  \varphi_{1} = \frac{A_{1-2}}{q} & \text{jei $\varphi_{2} = 0$}
\end{align*}

Matavimo vienetai:
\begin{align*}
  1 V = \frac{1 J}{1 C}
\end{align*}
Čia:
\begin{description}
  \item[$1 V$] – vienas voltas;
  \item[$1 J$] – vienas džaulis;
  \item[$1 C$] – vienas kulonas.
\end{description}

Vienas elektron-voltas:
\begin{align*}
  1 eV &= 1,6 \cdot 10^{-19} \cdot 1 &= 1,6 \cdot 10^{-19} J
\end{align*}

\section{Dielektrikų poliarizacija}

Dielektrikų rūšys:
\begin{description}
  \item[nepoliniai dielektrikai] – tie, kurių normaliomis sąlygomis
    elektronų svorio centras sutampa su elektrono svorio centru.
  \item[dipoliniai dielektrikai] – tie, kurių elektronų svorio centras
    ir branduolio svorio centras nesutampa.
\end{description}

Dipolinis momentas:
\begin{align*}
  \vec{p_{e}} &= q \cdot \vec{l}
\end{align*}

Indukuotą dipolinį elektrinį momentą proporcingą elektrinio lauko stipriui.

Joninė poliarizacija – 

Poliarizacijos vektorius:
\begin{align*}
  \vec{P}
  &= \frac{1}{\Delta V} \sum _{i=1} ^{h} \vec{P}_{ei} \\
  &= n_{0} \cdot \alpha \cdot \varepsilon_{0} \cdot \vec{E} =
    \Xi \varepsilon \vec{E_{0}}
\end{align*}

\begin{align*}
  E &= E_{0} - E' \\
  E' &= E_{0} - E = E_{0} - \frac{E_{0}}{\varepsilon} 
  = E_{0} \left( 1 - \frac{1}{\varepsilon} \right) \\
  E' &= (\varepsilon - 1)E \\
  \varepsilon = 1 + \Xi
\end{align*}
Čia:
\begin{description}
  \item[$E_{0}$] – išorinis laukas;
  \item[$E'$] – poliarizacijos laukas.
\end{description}

\subsection{Poliarizacijos histerezė}

Pjezo efektas – spaudžiant medžiagą, turinčią dipolių, dipoliai
orientuojasi ir medžiagos paviršiuje atsiranda krūviai.

Kristaliniai dialektrikai (dar vadinami signeto arba feroelektrikais) – 
prie tam tikros temperatūros pasižymi savaimine poliarizacija.

Histerizės kilpa –

\subsection{Kondensatorius}

Elektrinė talpa:
\begin{align*}
  C &= \frac{q}{\varphi} \\
  Q &= CU \\
  E &= \frac{\varphi_{1} - \varphi_{2}}{d} \\
  E &= \frac{\sigma}{\varepsilon_{0}\varepsilon} = 
  \frac{q}{\varepsilon_{0}\varepsilon S} \\
  \sigma \equiv D &= \varepsilon_{0} \varepsilon E \\
  c &= \frac{\varepsilon_{0} \varepsilon S}{d}.
\end{align*}
Čia:
\begin{description}
  \item[$Q$] – krūvis;
  \item[$C$] – talpa;
  \item[$U$] – potencialų skirtumas;
  \item[$d$] – atstumas tarp potencialų;
\end{description}

\subsection{Laidininkai elektriniame lauke}

Laidininko viduje nėra krūvio.
