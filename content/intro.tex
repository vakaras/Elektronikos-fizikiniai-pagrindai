\chapter{Įvadas}

\section{SI vienetų sistema}

Pagrindiniai fizikiniai vienetai:
\begin{description}
  \item[kandela] – šviesos šaltinio stiprumas;
  \item[metras] – ilgis;
  \item[kilogramas] – masė;
  \item[sekundė] – laikas;
  \item[kelvinas] – temperatūra;
  \item[molis] – medžiagos kiekis;
  \item[amperas] – srovės stiprio stiprumas.
\end{description}

Taip pat dažnai naudojami du papildomi vienetai:
\begin{description}
  \item[radianas] – plokščias kampas;
  \item[sferadianas] – erdvinis kampas.
\end{description}

\section{Dinamikos skyriaus priminimas}

Antrasis Niutono dėsnis:
\begin{equation*}
  \vec{a} = \frac{\vec{F}}{m}
\end{equation*}

\begin{exmp}
  Tarkime, turime elektroną, besisukantį apie protoną apskritimu, kurio
  spindulys yra $R$. Mums reikia rasti elektrono 
  įcentrinį pagreitį.

  Šis uždavinys realiai yra dinamikos uždavinys, nepaisant to, kad
  veikianti jėga yra elektrostatinės prigimties:
  \begin{align*}
    F &= \frac{|e| \cdot q}{4 \pi \varepsilon_{0} \varepsilon R^{2}} \\
    F &= m \cdot a_{\t{įc}} \\
    a_{\t{įc}}
    &= \frac{|e| \cdot q}{4 \pi \varepsilon_{0} \varepsilon m R^{2}} \\
  \end{align*}
  čia:
  \begin{description}
    \item[$F$] – jėga, verčianti elektroną judėti apskritimu;
    \item[$a_{\t{įc}}$] – elektrono įcentrinis pagreitis;
    \item[$m$] – elektrono masė;
    \item[$e$] – elektrono krūvis;
    \item[$q$] – protono krūvis;
    \item[$\varepsilon_{0}$] – elektrinė konstanta,
      $\varepsilon_{0} \approx $
      $8,85 \cdot 10^{-12} \frac{C^{2}}{N \cdot m^{2}}$
    \item[$\varepsilon$] – aplinkos dielektrinė skvarba;
    \item[$R$] – atstumas tarp protono ir elektrono.
  \end{description}
\end{exmp}
