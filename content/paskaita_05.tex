\subsection{Elektroninė metalų laidumo teorija}

Metaluose krūvio nešėjai yra elektronai. Pagrindimas: 

Elektronas, susidūręs su metalo gardelės jonu, gali pradėti judėti
bet kuria kryptimi ir po kiekvieno tokio susidūrimo dreifuojančio
elektrono pradinis greitis yra lygus nuliui. Dėl elektrinio lauko
poveikio, elektronas pradeda judėti jo veikimo kryptimi ir per laiką
$t$ vėl sutinka gardelės joną.

Elektrono inercijos jėga:
\begin{equation*}
  F_{\t{in}} = -ma
\end{equation*}
čia:
\begin{description}
  \item[$m$] – masė;
  \item[$a$] – pagreitis.
\end{description}

Elektros šaltinio kuriama elektros varomoji jėga:
\begin{align}
  E &= \frac{A_{\t{paš}}}{q} \\
  \intertext{Kadangi darbas yra lygus jėgos ir kelio sandaugai:}
  &= \frac{F_{\t{in}} l}{q} \\
  &= -\frac{mal}{q} \\
  \intertext{Pagal apibrėžimą, pagreitis yra greičio pokytis per laiko
  vienetą. Kadangi, pas mus pradinis greitis yra 0, tai gauname:}
  &= -\frac{mvl}{qt} \label{eq:metalu_laidumo_1} \\
\end{align}
čia:
\begin{description}
  \item[$A_{\t{paš}}$] – pašalinių jėgų darbas;
  \item[$q$] – pratekėjęs krūvis;
  \item[$F_{\t{in}}$] – inercijos jėga;
  \item[$l$] – laidininko ilgis.
  \item[$v$] – elektrono per laiką $t$ pasiektas greitis;
  \item[$t$] – po kiek laiko nuo judėjimo pradžios elektronas sutiko
    joną.
\end{description}

Krūvis, kuris prateka laidininku:
\begin{equation}
  Q = I t \label{eq:metalu_laidumo_3}
\end{equation}
čia:
\begin{description}
  \item[$I$] – srovės stipris.
\end{description}

Iš Omo dėsnio:
\begin{align}
  I &= \frac{E}{R} \\
  \intertext{Įsistatome \ref{eq:metalu_laidumo_1} ir gauname:}
  &= -\frac{mal}{Rq} \label{eq:metalu_laidumo_2} \\
\end{align}
čia:
\begin{description}
  \item[$R$] – laidininkų ir galvanometro varža.
\end{description}

Sulyginę \ref{eq:metalu_laidumo_2} ir \ref{eq:metalu_laidumo_3}
gauname:
\begin{align*}
  I &= \frac{mlv_{0}}{qRt} \\
  &= \frac{Q}{t} \\
  \frac{q}{m}
  &= \frac{v_{0}l}{QR} \\
  &= (\t{nuo }1,5 \t{ iki } 1,6) \cdot 10^{11} \tfrac{C}{kg}
\end{align*}
čia:
\begin{description}
  \item[$\frac{q}{m}$] – savitasis dalelės krūvis.
\end{description}

Gautąjį savitąjį dalelės krūvį palyginę su elektrono:
\begin{align*}
  \frac{e}{m} &= 1,76 \cdot 10^{11} \tfrac{C}{kg}.
\end{align*}
galime teigti, jog krūvio nešėjai yra elektronai.

TODO: Suprasti ir perrašyti.

\begin{align*}
  \frac{mv^{2}}{2} &= \frac{3}{2} k T
\end{align*}

\begin{align*}
  v_{kv} &= 1,1 \cdot 10^{5} \tfrac{m}{s}
\end{align*}

Tvarkingo judėjimo greitis:
\begin{align*}
  v &\equiv 8 \cdot 10^{-4} \tfrac{m}{s}
\end{align*}

Vidutinis elektrono greitis (?):
\begin{align*}
  \t{ū} &= 10^{6} \tfrac{m}{s}
\end{align*}

Elektrono greitis:
\begin{align*}
  v &= \mu E
\end{align*}

\section{Srovės magnetinis laukas}

Magnetinė indukcija:
\begin{align*}
  \vec{B}
\end{align*}

% INCLUDE: Brėžinį iš Audriaus sąsiuvinio. ID=#0003

\begin{align*}
  \frac{F_{m}}{|q|v} &= B \t{konstanta}
\end{align*}

Lorenco jėga:
\begin{align*}
  \vec{F_{m}} &= q [\vec{v} \cdot \vec{B}] \\
  F_{m} &= |q|vB \sin \alpha
\end{align*}

Jėga elektromagnetiniame lauke:
\begin{align*}
  F &= q \vec{E} + q \left[ \vec{v} \cdot \vec{B} \right]
\end{align*}

Ampero jėga:
\begin{align*}
  d \vec{F} &= I \left[ d\vec{l} \cdot B \right]
\end{align*}

\begin{align*}
  d \vec{B} &= \frac{\mu_{0}}{4\pi} \frac{I [d \vec{l}\vec{r}]}{r^{3}} \\
  dB &= \frac{\mu_{0}}{4\pi} \frac{I\cdot dl \sin \alpha}{r^{2}} \\
  B &= \frac{\mu_{0}I}{4\pi}
    \int _{(l)} \frac{\left[ d\vec{l} \vec{r} \right]}{r^{3}}
\end{align*}

\begin{align*}
  \mu_{0} &= 4 \pi \cdot 10^{-7} \tfrac{Tm}{A} \\
  &= 4 \pi \cdot 10 ^{-7} \tfrac{H}{m}
\end{align*}

\begin{align*}
  r &= \frac{b}{\sin \alpha} \\
  dl &= \frac{r \cdot d \alpha}{\sin \alpha}
  &= \frac{b d \alpha}{\sin ^{2} \alpha}
\end{align*}

\begin{align*}
  dB &= \frac{\mu_{0}}{4\pi}
    \frac{I \cdot b \cdot d\alpha \sin \alpha \sin ^{2} \alpha}{
      b^{2} \cdot \sin ^{2} \alpha} \\
  \int dB &= \int _{0} ^{\pi} \frac{\mu_{0}}{4\pi} 
    \frac{I}{b} \sin \alpha d \alpha \\
  B = \frac{\mu_{0}}{4\pi} \frac{2I}{b}
\end{align*}

\begin{exmp}
  % „Kirchofo metodas uzd.pdf“, 2 psl.

  Pirma Kirchofo taisyklė mazgui B:
  \begin{align*}
    I_{1} + &I_{2} + &I_{3} - &I_{4} &= 0
  \end{align*}

  Antra Kirchofo taisyklė:
  \begin{align*}
    (r_{1} + R_{1}) I_{1} - (R_{2} + r_{2}) &I_{2} && &= E_{1} - E_{2} \\
    (R_{2} + r_{2}) &I_{2} - R_{3}&I_{3} & &= E_{2} \\
    &&R_{3}I_{3} + R_{4}&I_{4} &= 0
  \end{align*}

  Sprendžiam lygčių sistemą Gauso metodu:
  \begin{align*}
    1 & 1 & 1 & -1 & 0 \\
    2 & -4 & 0 & 0 & 6 \\
    0 & 4 & -4 & 0 & 4 \\
    0 & 0 & 4 & 2 & 0 \\
  \end{align*}
  \begin{align*}
    1 & 1 & 1 & -1 & 0 \\
    0 & -6 & -2 & 2 & 6 \\
    0 & 4 & -4 & 0 & 4 \\
    0 & 0 & 4 & 2 & 0 \\
  \end{align*}
  \begin{align*}
    1 & 1 & 1 & -1 & 0 \\
    0 & -6 & -2 & 2 & 6 \\
    0 & 0 & -4 & 1 & 6 \\
    0 & 0 & 4 & 2 & 0 \\
  \end{align*}
  \begin{align*}
    1 & 1 & 1 & -1 & 0 \\
    0 & -3 & -1 & 1 & 3 \\
    0 & 0 & 0 & 1 & 2 \\
    0 & 0 & 2 & 1 & 0 \\
  \end{align*}

  \begin{align*}
    I_{4} &= 2 \t{A} \\
    I_{3} &= -1 \t{A} \\
    I_{2} &= 0 \t{A} \\
    I_{1} &= 3 \t{A}
  \end{align*}
\end{exmp}

\begin{exmp}
  % „Kirchhofo grandine.jpg“

  Pirma Kirchofo taisyklė mazgams A, B ir C:
  \begin{align*}
    %&I_{1} + &I_{2} + &I_{3} + &I_{4} + &I_{5} + &I_{6} &= 0
    \t{A:}&  & + &I_{3} - &I_{4} + &I_{5} & &= 0 \\
    \t{B:} &I_{1} - &I_{2} - &I_{3} &&& &= 0\\
    \t{C:} &&&  &I_{4} - &I_{5} + &I_{6} &= 0
  \end{align*}

  Antra Kirchofo taisyklė:
  \begin{align*}
    R_{5}-R_{4}&: &&& (R_{4} + r_{3}) &I_{4} + R_{5}&I_{5} & &= E_{3} \\
    R_{5}-R_{3}-R_{2}-E_{2}&:
      & R_{2} &I_{2} - R_{3}&I_{3} + R_{5} &I_{5} + r_{2} &I_{6} &= E_{2} \\
    R_{4}-R_{3}-R_{1}-E_{1}-E_{2}&:
      -(r_{1} + R_{1}) &I_{1} & - R_{3} &I_{3} -
      (r_{3} + R_{4})&I_{4} + &r_{2} &I_{6} = -E_{1} + E_{2}
  \end{align*}
  \begin{align*}
    &:                  &             &              &        (R_{4} + r_{3}) &I_{4}  + R_{5}&I_{5} & &= E_{3} \\
    &:                  &       R_{2} &I_{2}  - R_{3}&I_{3}                           + R_{5}&I_{5} + r_{2} &I_{6} &= E_{2} \\
    &: -(r_{1} + R_{1}) &I_{1}        &       - R_{3}&I_{3} - (r_{3} + R_{4})&I_{4} + &r_{2} &I_{6} = -E_{1} + E_{2}
  \end{align*}

  % Trečia su klaidomis.
  
\end{exmp}
