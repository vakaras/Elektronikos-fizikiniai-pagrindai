\section{Kietųjų kūnų juostinis modelis}

\subsection{Kontaktinis potencialų skirtumas}

Greičiausiai judantys elektronai nuolat išlekia į aplinką ir arti
laidininko paviršiaus (per kelis tarpatominius atstumus) debesėlį.
Dėl to metale atsiranda teigiamų krūvių, kurie su debesėliu sudaro
dvigubą elektrinį sluoksnį, primenantį plokščią kondensatorių.
Išlėkdamas iš metalo į vakuumą, elektronas atlieka vadinamąjį
išlaisvinimo darbą (turi įveikti dvigubo sluoksnio elektrinį
lauką). Potencialų skirtumas dvigubo sluoksnio elektriniame lauke
$\Delta \varphi$ vadinamas kontaktiniu potencialų skirtumu (KPS)
tarp metalo ir gaubiančios aplinkos:
\begin{equation*}
  \Delta \varphi = \frac{A}{|e|}
\end{equation*}
Čia darbas $A$ matuojamas $eV$. $A$ svyruoja kelių $eV$ ribose ir
priklauso nuo metalo prigimties, temperatūros, paviršiaus būsenos.

% INCLUDE brėžinys.

Sujungus du skirtingus metalus analogiškai susidaro dvigubas ribinis
įelektrintas sluoksnis ir metalai įgauna skirtingus potencialus
$\varphi_{1}$ ir $\varphi_{2}$
($\Delta \varphi = \varphi_{1} - \varphi_{2}$). Aleksandras Volta
(itališkai Alessandro Giuseppe Antonio Anastasio Volta) 1785 metais
nustatė du dėsnius.

\begin{defn}[Pirmas kontaktinio potencialų skirtumo dėsnis]
  Tarp dviejų skirtingų metalų idealiu atveju KPS priklauso tik nuo
  metalų prigimties ir temperatūros.
\end{defn}

\begin{defn}[Antras kontaktinio potencialų skirtumo dėsnis]
  Nuosekliai sujungus skirtingų metalų laidininkus, kurių temperatūra
  vienoda, KPS tarp jų galų nepriklauso nuo tarpinių ir yra lygi
  kraštinių KPS.
\end{defn}

Priežastys dėl ko atsiranda KPS:
\begin{enumerate}
  \item skirtingas išlaisvinimo darbas
    $\Delta \varphi_{12}' = \frac{|A_{1}| - |A_{2}|}{l}$
    ($\Delta \varphi' \sim 1 V$) – ši priežastis vadinama vidiniu KPS;
  \item skirtingos laisvų elektronų koncentracijos (jei $n_{1} > n_{2}$,
    tai iš metalo turinčio $n_{1}$ daugiau difunduos į medžiagą
    turinčią $n_{2}$)
  \begin{equation*}
    \Delta \varphi''_{12} = \frac{kT}{e}\ln\frac{n_{1}}{n_{2}},
  \end{equation*}
  čia:
  \begin{description}
    \item[$k$] – Bolcmano konstanta;
    \item[$T$] – absoliutinė temperatūra.
  \end{description}
  (Ši priežastis vadinama vidine KPS.)
\end{enumerate}

\subsection{Termoelementai}

TODO: Sutvarkyti šį skyrelį pagal dėstytojo konspektą.

Panagrinėkime atvejį, kai $n_{1} > n_{2}$ ir $T_{a} > T_{b}$.
\begin{align*}
  E_{T}
  &= \Delta \varphi_{12a}'' + \Delta \varphi_{12b} \\
  &= \frac{kT_{a}}{e}\ln\frac{n_{1}}{n_{2}} +
    \frac{kT_{b}}{e}\ln\frac{n_{2}}{n_{1}} \\
  &= \frac{k}{l} \left( T_{a} - T_{b} \right) \ln \frac{n_{1}}{n_{2}} \\
  &= \alpha (T_{a} - T_{b}) \\
\end{align*}
čia:
\begin{description}
  \item[$E_{T}$] – termo elektrovara;
  \item[$\alpha$] – savitoji termo elektrovara, kuri susidaro esant
    1 laipsnio temperatūrų skirtumui.
\end{description}

Kai $\Delta T \equiv 100^{o}C$, tai $E_{T} ~ (1-4)mV$.

Vienas iš pritaikymų – termometrų gamyba. Kitas – Pentjė reiškinys
šaldytuvų gamybai.

\subsection{Kietų kūnų juostinės teorijos modelis}

TODO: Brėžinys.

Elektrono energija atome yra kvantuota. Pagal Volfgango Ernsto Paulio
(vokiškai Wolfgang Ernst Pauli) draudimo principą viename lygmenyje
gali būti ne daugiau, kaip du elektronai su skirtingais
(antilygiagrečiais) sukiniais.

Atomams suartėjus, jie vienas kitą veikia ir pasikeičia jų elektronų
energijos lygmenys. Atsiranda $N$ artimų, bet jau nesutampančių lygmenų,
energijos lygmuo suskyla į $N$ lygmenų ir sudaro juostą (zoną).
Labiausiai suskyla valentinių elektronų ir aukštesni lygmenys bei
skildami sudaro leistiną elektronų energijos juostą (kelių $eV$ pločio).
Tarpai (pažymėkime juos $B$) tarp leistinų energijos juostų (pažymėkime
jas $A$) vadinami draustinėmis energijos juostomis. $B$ plotis gali
būti nuo dešimtųjų $eV$ iki kelių $eV$. Valentiniais elektronais
užpildyta juosta (visiškai arba dalinai) vadinama valentine juosta.
Gretima aukštesnės energijos juosta (tuščia, arba dalinai užpildyta
elektronais) vadinama laidumo juosta.

Metaluose laidumo ir valentinė juostos persidengia, todėl elektronams
lengva pereiti iš valentinės juostos į laidumo juostą. Puslaidininkiuose
valentinė juosta yra visiškai užpildyta ir elektronas be papildomos
energijos negali patekti į laidumo juostą. Jei šiluminio laidumo
energijos nepakanka, kad elektronas pereitų iš valentinės juostos į
laidumo, tai medžiaga yra jau ne puslaidininkis, o dielektrikas.

\subsection{Puslaidininkiai}

Puslaidininkių savitoji varža gali būti nuo $10^{-5} \Omega \cdot m$
iki $10^{8} \Omega \cdot m$ ir skirtingai nei metaluose ji kylant
temperatūrai mažėja.

Kai kurių puslaidininkių aktyvacijos energija:
\begin{itemize}
  \item $Sb$ – $0,12 eV$,
  \item $Te$ – $0,36 eV$,
  \item $Ge$ – $0,75 eV$,
  \item $Si$ – $1,1 eV$,
  \item $As$ – $1,2 eV$,
  \item $P$  – $1,5 eV$,
  \item $Se$ – $1,7 eV$,
  \item $S$  – $2,5 eV$.
\end{itemize}
Taip pat puslaidininkiams priklauso daug junginių: $Cu_{2}O, PbS, CdSe$,
$As_{2}Se_{3},\ldots,CdS$.

Savasis puslaidininkių laidumas gali būti dvejopas: elektroninis ir
skylinis. Su laidumu susijusios dvi sąvokos:
\begin{description}
  \item[elektronų generacija] – elektronų perėjimas iš valentinės
    juostos į laisvąją;
  \item[elektronų rekombinacija] – elektronų perėjimas iš
    laisvosios juostos į valentinę.
\end{description}

TODO: Brėžinys.

\subsubsection{Priemaišinis laidumas}

Be savojo puslaidininkinio laidumo, dar yra ir priemaišinis.
Puslaidininkiai, kuriuose priemaišos atomams sudarius ryšį su
puslaidininkio atomais lieka nesurištų elektronų, vadinami
elektroninio laidumo, arba kitaip $n$ tipo puslaidininkiais.
Atitinkamai priemaišos vadinamos donorinėmis. Pavyzdžiui,
jei germanio, kuris turi keturis valentinius elektronus, atomą
pakeisime penkiavalenčiu fosforo atomu, tai 4 jo valentiniai
elektronai bus surišti su gretimais germanio atomais, o penktas
elektronas ryšyje nedalyvaus – bus silpnai surištas su branduoliu
ir lengvai galės pereiti į laidumo juostą. Šis penktojo elektrono
lygmuo yra draustinėje juostoje arčiau laidumo juostos ir yra
vadinamas donoriniu lygmeniu. TODO: Brėžinys.

Atitinkamai puslaidininkai, kuriuose priemaišos atomams sudarant
ryšį su puslaidininkio atomais trūksta elektronų, vadinami skylinio
laidumo, arba kitaip $p$ tipo puslaidininkiais. Šiuo atveju
priemaišos vadinamos akceptorinėmis. Pavyzdžiui, jei germanio
atomą pakeisime trivalenčiu boro atomu, tai jam kovalentinio
ryšio sudarymo truks vieno elektrono, kurį jis pasiims iš gretimo
germanio atomo, kuriame tuo tarpu atsiras „teigiama“ skylė.
Akceptoriniai – elektronų neužpildyti nauji energijos lygmenys –
išsidėsto arčiau valentinės juostos draudžiamų energijų juostoje.
Iš užpildytos valentinės juostos elektronai lengvai peršoka
į akceptorinį lygmenį, o valentinėje juostoje atsiranda skylė.
Taigi valentinė juosta tampa skylinio laidumo juosta.

\subsubsection{Puslaidininkių laidumo priklausomybė nuo temperatūros}

Grynas puslaidininkis $0K$ temperatūroje yra izoliatorius. Metalų
elektrinis laidumas, pakėlus temperatūra 1 laipsniu $[273K;373K]$
intervale, sumažėja nuo $0,3 \%$ iki $0,4\%$, o puslaidininkių
padidėja nuo $3\%$ iki $6\%$ kiekvienam laipsniui.

Puslaidininkinis termoelementas. TODO: Brėžinys. Tarkime, kad turime $n$
tipo puslaidininkį, kurio vieną galą kaitiname. Kaitinamoje vietoje
atsiranda laisvų elektronų, kurie difunduoja į šaltesnįjį galą, nes
jame yra mažiau laisvų elektronų. Šildomas galas įsielektrina $+$,
o nešildomas $-$ – atsiranda termo evj
$E = \alpha (T_{\t{k}} - T_{\t{š}})$.
\begin{note}
  Kaitinant $p$ puslaidininkį (skylinį) jis įsielektrins priešingai.
\end{note}
Metaluose tokio efekto negauname, nes elektronų koncentracija juostose
iš esmės nepriklauso nuo temperatūros.

TODO: Papildyti iš konspekto.
