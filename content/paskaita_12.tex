
% INCLUDE brėžinys.

Išėjimo darbas (išlaisvinimo darbas) – jį atlieka elektronas keliaudamas
iš metalo „į laisvę“.

Kontaktinis potencialų skirtumas (KPS): $\Delta \varphi$. (Potencialų
skirtumas tarp išlaisvintų elektronų ir medžiagoje atsiradusių jonų.)

\begin{align*}
  \Delta \varphi &= \frac{A}{|e|} \\
\end{align*}
čia:
\begin{description}
  \item[$A$] – išlaisvinimo darbas, matuojamas $eV$.
\end{description}

\begin{defn}[Pirmas kontaktinio potencialų skirtumo dėsnis]
  Tarp dviejų skirtingų metalų idealiu atveju KPS priklauso tik nuo
  metalų prigimties ir temperatūros.
\end{defn}

\begin{defn}[Antras kontaktinio potencialų skirtumo dėsnis]
  Nuosekliai sujungus skirtingų metalų laidininkus, kurių temperatūra
  vienoda, KPS tarp jų galų nepriklauso nuo tarpinių ir yra lygi
  kraštinių KPS.
\end{defn}

Priežastys dėl ko atsiranda KPS:
\begin{enumerate}
  \item skirtingas išlaisvinimo darbas
    $\Delta \varphi_{12}' = \frac{|A_{1}| - |A_{2}|}{l}$
    ($\Delta \varphi' ~ 1 V$);
  \item skirtingos laisvų elektronų koncentracijos (jei $n_{1} > n_{2}$,
    tai iš metalo turinčio $n_{1}$ daugiau difunduos į medžiagą
    turinčią $n_{2}$)
  \begin{equation*}
    \Delta \varphi''_{12} = \frac{kT}{e}\ln\frac{n_{1}}{n_{2}}
  \end{equation*}
  ;
\end{enumerate}

\subsection{Termoelementai}

Panagrinėkime atvejį, kai $n_{1} > n_{2}$ ir $T_{a} > T_{b}$.
\begin{align*}
  E_{T}
  &= \Delta \varphi_{12a}'' + \Delta \varphi_{12b} \\
  &= \frac{kT_{a}}{e}\ln\frac{n_{1}}{n_{2}} +
    \frac{kT_{b}}{e}\ln\frac{n_{2}}{n_{1}} \\
  &= \frac{k}{l} \left( T_{a} - T_{b} \right) \ln \frac{n_{1}}{n_{2}} \\
  &= \alpha (T_{a} - T_{b}) \\
\end{align*}
čia:
\begin{description}
  \item[$E_{T}$] – termo elektrovara;
  \item[$\alpha$] – savitoji termo elektrovara, kuri susidaro esant
    1 laipsnio temperatūrų skirtumui.
\end{description}

Kai $\Delta T \equiv 100^{o}C$, tai $E_{T} ~ (1-4)mV$.

Vienas iš pritaikymų – termometrų gamyba. Kitas – Pentjė reiškinys
šaldytuvų gamybai.

\subsection{Kietų kūnų juostinės teorijos modelis}

Pagal Pouli principą…

Elektronas gali būti tik tam tikrose orbitose (lygmenyje).

Pažymėkime juostas raide $A$, o tarpus – $B$. $A$ – leidžiamos energijos
juostos, o $B$ – draudžiamos energijos juostos.

Valentinė juosta – juosta, kurioje yra valentiniai elektronai.

Metaluose laidumo ir valentinė juostos persidengia.

Jei šiluminio laidumo energijos nepakanka, kad elektronas pereitų
iš valentinės juostos į laidumo, tai medžiaga yra jau ne puslaidininkis,
o dielektrikas.

Puslaidininkiai būna elektroninio laidumo ir skylinio laidumo.

Elektronų generacija – elektronų perėjimas iš valentinės juostos į
laisvąją.

Elektronų rekombinacija – elektronų perėjimas iš laisvosios juostos į
valentinę.

Laidumas gali būti savasis (silicis, germanis…) ir priemaišinis.

Silicio kristalas su fosforo (donorinėmis) priemaišomis – n tipo
puslaidininkis.

Silicio kristalas su boro (akceptorinėmis) priemaišomis – p tipo
puslaidininkis.

Puslaidininkių laidumo priklausomybė nuo temperatūros:
\begin{align*}
  \vec{j} &= \underbrace{e n \mu}_{\sigma} \vec{E} \\
\end{align*}

Puslaidininkinis termo elementas. Kaitinant galą $n_{1} > n_{2}$, todėl
difusijos elektronai lekia į ten kur mažesnis tankis (ten kur šalčiau).
