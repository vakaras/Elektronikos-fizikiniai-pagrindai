\begin{align*}
  j &=& \frac{d \sigma}{dt} \\
\end{align*}
čia:
\begin{description}
  \item[$j$] – srovės tankis;
  \item[$\sigma$] – paviršinio krūvio tankis;
\end{description}

\begin{align*}
  \sigma &\equiv D \\
  D
  &=& \varepsilon \varepsilon_{0} E \\
  &=& \frac{Q}{4 \pi r^2} \\
\end{align*}

Slinkties srovė:
\begin{align*}
  j_{sl}
  &=& \frac{\delta D}{\delta t} \\
  &=& \frac{d\sigma}{dt} \\
  &=& \varepsilon \varepsilon_{0} \frac{\delta E}{\delta t} \\
\end{align*}

Talpinė kondensatoriaus varža:
\begin{align*}
  X_{c} &=& \frac{1}{w C} \\
\end{align*}

\section{Varžų analizė (?)}

Kai į grandinę įjungta paprasta varža:
\begin{align*}
  E &=& E_{0} \sin wt \\
  I &=& \frac{E}{R} \\
  &=& \underbrace{\frac{E_{0}}{R}}_{I_{0}} \sin wt \\
\end{align*}
\begin{verbatim}
     ----
-----|  |------
|    ----     |
|             |
|             |
----- ~ -------
\end{verbatim}

Kai į grandinę įjungtas kondensatorius:
\begin{align*}
  E &=& E_{0} \sin wt \\
  E &=& \frac{Q}{C} \\
  \frac{dE}{dt} &=& \frac{1}{C} \frac{dQ}{dt} \\
  \intertext{Srovės apibrėžimas bendru atveju:}
  I &=& \frac{dQ}{dt} \\
  \intertext{Viską susistatę gauname:}
  \underbrace{w E_{0} C}_{\t{aplitudė} I_{0}}
    \cos \underbrace{w t}_{\t{fazė}} &=& I \\
  I &=& I_{0} \cos w t \\
  \cos wt &=& \sin \left( wt + \frac{\pi}{2} \right) \\
  I_{0} &=& w C E_{0} \\
  I &=&  I_{0} \sin \left( wt + \frac{\pi}{2} \right) \\
  E_{0} &=& \underbrace{\frac{1}{wC}}_{X_{C}} I_{0} \\
  \intertext{Primenama talpinė varža:}
  X_{C} &=& \frac{1}{wC} \\
\end{align*}
\begin{verbatim}
------||-------
|             |
|             |
----- ~ -------
\end{verbatim}

Kai į grandinę įjungta ritė:
\begin{align*}
  E &=& -L \frac{dI}{dt} \\
  E_{prd} &=& E \\
  L \frac{dI}{dt} &=& E_{0} \sin wt \\
  dI &=& \frac{E_{0}}{L} \sin wt \cdot dt \\
  I &=& \frac{E_{0}}{L} \int \sin wt dt \\
  \cos wt &=& -\sin \left( wt - \frac{\pi}{2} \right) \\
  I &=& - \underbrace{\frac{E}{w L}}_{I_{0}} \cos wt \\
  I &=& I_{0} \sin \left(wt - \frac{\pi}{2}\right) \\
  E_{0} &=& \underbrace{wL}_{X_{L}} I_{0} \\
\end{align*}
\begin{verbatim}
----oooooo-----
|             |
|             |
----- ~ -------
\end{verbatim}

Kai į grandinę įjungta paprasta varža, kondensatorius ir ritė:
\begin{align*}
  e
  &=& e_{R} + e_{C} + e_{L} \\
  &=& I_{0} R \sin wt + I_{0} X_{C} \sin \left( wt - \frac{\pi}{2} \right)
    + I_{0} X_{L} \sin \left(wt + \frac{\pi}{2}\right) \\
  &=& I_{0}R \sin wt + I_{0} X_{L} \cos wt - I_{0} X_{C} \cos wt \\
  \frac{e}{I_{0}} &=& R \sin wt + (X_{L} - X_{C})\cos wt \\
  \frac{e}{I_{0}Z}
  &=& \frac{R}{Z} \sin wt + \frac{X_{L} + X_{C}}{Z} \cos wt \\
  e &=& I_{0} Z
    \left( \cos \varphi \sin wt + \sin \varphi \cos wt \right) \\
\end{align*}
\begin{verbatim}
     ---┓
-----|  |---||---ooooo---
|    └──┘               |
|     R      C    L     |
|                       |
----------- ~ -----------
\end{verbatim}

Žr. 2.2.1 „Sinusinių dydžių vaizdavimas vektoriais“ skyrelį
Masioko vadovėlyje.

\begin{verbatim}
   /|
Z / |
 /  | (X_{L} - X_{C})
/___|
  R
\end{verbatim}

\begin{align*}
  \underbrace{Z}_{\frac{E_{0}}{I_{0}}}
    &=& \sqrt{R^{2} + \left( X_{L} - X_{C} \right)^{2}} \\
  \varphi &=& \arctan \frac{X_{L} - X_{C}}{R} \\
  \cos \varphi &=& \frac{R}{Z} \\
\end{align*}

$Z$ – atstojamoji grandinės varža, impedancas.

\begin{align*}
  Z &=& \sqrt{R^{2} + \left( wL - \frac{1}{wC} \right)^{2}}
\end{align*}

% INCLUDE: Brėžinį iš Audriaus sąsiuvinio. ID=#0006
\begin{align*}
  w_{0} L - \frac{1}{w_{0}C} &=& 0 \\
  w_{0} &=& \frac{1}{\sqrt{LC}} \\
  w_{0} &=& 2 \pi \frac{1}{T} \\
  \intertext{Tomsono formulė:}
  T = 2 \pi \sqrt{LC} \\
\end{align*}
čia:
\begin{description}
  \item[$w_{0}$] – rezonancinis dažnis.
\end{description}

\section{Kintamos srovės galia}

Momentinė galia:
\begin{align*}
  p &=& e \cdot i \\
  i &=& I_{0} \sin wt \\
  e &=& E_{0} \sin \left( wt + \varphi \right) \\
  p &=& E_{0} I_{0} \sin \left( wt + \varphi \right) \sin wt \\
\end{align*}

Efektinė:
\begin{align*}
  E_{0}
  &=& \underbrace{E_{ef}}_{\t{Dažniausiai žymime tiesiog} E} \sqrt{2} \\
  I_{0} &=& I_{ef} \sqrt{2} \\
  \sin \alpha \cdot \sin \beta
  &=& \frac{1}{2} \cos(\alpha - \beta) - \frac{1}{2} \cos(\alpha + \beta) \\
  \sin (wt + \varphi) \cdot \sin wt
  &=& \frac{1}{2} \cos \varphi - \frac{1}{2} \cos (2 wt + \varphi) \\
  E_{0}I_{0} &=&  2 UI \\
  p &=& UI \cos \varphi - UI\cos (2wt + \varphi) \\
  \intertext{vidutinė galia per periodą:}
  p
  &=& \frac{1}{T} \int _{0} ^{T} p dt \\
  &=& UI \cos \varphi \\
\end{align*}

\section{Pratybos}

% INCLUDE: Brėžinį iš Ramintos sąsiuvinio.

\emph{Sąlyga} Elektronas be pradinio greičio pralekia 10000 V potencialų
skirtumą ir įlekia į plokščią kondensatorių, lygiagrečiai jo plokštelėms.
Kondensatorius įelektrintas iki 100 V. Atstumas tarp kondensatoriaus
plokštelių yra 2 cm, o ilgis 20 cm. Rasti atstumą BC arba $h_{1} + h_{2}$
ekrane, nutolusiame nuo kondensatoriaus 1 metru.

\begin{align*}
  U_{0} &=& 10^{4} V \\ 
  U_{1} &=& 100 V \\
  d &=& 0,02 m \\
  l_{1} &=& 0,2 m \\
  l_{2} &=& 1m \\
  \intertext{Elektrono kinetinė energija yra lygi elektrinio lauko darbui:}
  \frac{m v_{0}^{2}}{2} &=& e U_{0} \\
  \intertext{Iš čia gauname, kad elektrono pradinis greitis būtų:}
  v_{0}
  &=& \sqrt{\frac{2 e U_{0}}{m}} \\
  &=& 0,593 \cdot 10^{8} \frac{m}{s} \\
  \intertext{Kondensatoriaus viduje elektroną veikia jėga:}
  F
  &=& e \cdot E \\
  &=& e \frac{U_{1}}{d} \\
  \intertext{Elektronas tarp plokštelių skries:}
  l_{1} &=& v_{0} \cdot t' \\
  t' &=& \frac{l_{1}}{v_{0}} \\
  \intertext{Elektrono greičio projekcija ekrano atžvilgiu:}
  v_{y}
  &=& a t' \\
  &=& \frac{F}{m} \cdot \frac{l_{1}}{v_{0}} \\
  &=& \frac{eU_{1}}{dm} \frac{l_{1}}{v_{0}} \\
  \intertext{Dabar galime apskaičiuoti atstumą $h_{1}$:}
  h_{1}
  &=& \frac{a t'^{2}}{2} \\
  &=& \frac{eU_{1}}{dm} \left( \frac{l_{1}}{v_{0}} \right)^{2} \\
  \intertext{Pastebime proporciją:}
  \frac{l_{2}}{h_{2}} &=& \frac{v_{0}}{v_{y}} \\
  \intertext{Dabar galime išsireikšti $h_{2}$:}
  h_{2}
  &=& \frac{v_{y} l_{2}^{2}}{v_{0}} \\
  &=& \frac{eU_{1} l_{2}^{2}l_{1}}{dmv_{0}^{2}} \\
  \intertext{Viską sudėją:}
  h_{1} + h_{2}
  &=& \frac{eU_{1}}{dm} \left( \frac{l_{1}}{v_{0}} \right)^{2} +
    \frac{eU_{1} l_{2}^{2}l_{1}}{dmv_{0}^{2}} \\
  &=& \frac{eU_{1}l_{1}}{dmv_{0}}
    \left( l_{1} + \frac{l_2^2}{v_{0}} \right) \\
  &=& 0,06 m
\end{align*}

%Python skaičiavimas:
%from math import sqrt

%u0 = 10 **4
%u1 = 100
%d = 0.02
%l1 = 0.2
%l2 = 1
%e = 1.6021892 * 10 ** (-19)
%m = 9.1 * 10 ** (-31)
%v0 = 0.593 * 10 ** 8
%v0 = sqrt((2 * e * u0)/(m))

%h1 = ((e * u1)/(d * m)) * ((l1)/(v0))**2
%h2 = (e * u1 * l2 ** 2 * l1)/(d * m * v0 ** 2)

%print h1+ h2
