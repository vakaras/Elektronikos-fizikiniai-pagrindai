\section{Harmoninio svyravimo apibrėžimas}

TODO: Išsiaiškinti, ką iš tiesų reikėtų žinoti.

Harmoninio svyravimo lygtis:
\begin{equation}
  x = A \sin \left( \omega_{0}t + \alpha \right)
  \label{eq:harmonine_x}
\end{equation}
čia:
\begin{description}
  \item[$x$] – atsilenkimas (poslinkis) nuo pusiausvyros padėties
    (pavyzdžiui, materialaus taško);
  \item[$\omega_{0}$] – nuosavų svyravimų dažnis;
  \item[$\alpha$] – pradinė fazė;
  \item[$t$] – laiko momentas.
\end{description}

Raskime greitį:
\begin{align*}
  v_{x}
    &= x' \\
    &= A \omega_{0} \cos \left( \omega_{0}t + \alpha \right) \\
    &= A \omega_{0}
        \sin \left( \omega_{0}t + \alpha + \frac{\pi}{2} \right) \\
\end{align*}
Gavome, kad greitis aplenkia poslinkį per $\frac{\pi}{2}$. Raskime
pagreitį:
\begin{align}
  a_{x}
    &= x'' \\
    &= -A \omega_{0}^{2} \sin \left( \omega_{0}t + \alpha \right) \\
    &= A \omega_{0}^{2} \sin \left( \omega_{0}t + \alpha + \pi \right) 
  \label{eq:harmonine_a} \\
\end{align}
Gavome, kad pagreitis ir poslinkis yra priešingų fazių. Padalinę
\ref{eq:harmonine_a} iš \ref{eq:harmonine_x} gauname harmoninio svyravimo
diferencialinę lygtį:
\begin{equation*}
  x'' + \omega_{0}^{2}x = 0
\end{equation*}

\subsection{LC kontūras ir elektromagnetiniai svyravimai}

Kiekviena elektrinė grandinė (be šaltinio) gali turėti varžą, induktyvumą
ir talpą. Nagrinėsime idealų atvejį, kai ritė neturi aktyvinės varžos.

Tegul kondensatorius įelektrintas $+Q_{0}, -Q_{0}$. Laiko momentu $t=0$
grandinę sujungiame. Išsielektrinant kondensatoriui srovė ritėje stiprėja.
Kiekvienu laiko momentu potencialų skirtumas kondensatoriuje
($U = \frac{Q}{C}$) yra lygus potencialų skirtumui ritėje, tai yra
saviindukcijos evj. ($-L\frac{dI}{dt}$), nes nėra įtampos kritimo
aktyvinėje varžoje.

Kai kondensatorius išsielektrina ($Q = 0$), srovė ritėje pasiekia
didžiausią vertę, indukcija ritėje yra taip pat maksimali. Energija,
sukaupta kondensatoriuje, virto ritės magnetinio lauko energija:
\begin{align*}
  W_{e} &= W_{m} \\
  \frac{CU_{C}^{2}}{2} &= \frac{LI_{m}^{2}}{2} \\
\end{align*}
Toliau srovė ritėje mažėja ir įelektrina kondensatorių. Kai $I = 0$, 
tai $Q = Q_{0}$, tik dabar kondensatorius įelektrintas priešingu
poliariškumu, negu buvo pradžioje ($t = 0$). Procesas vyks toliau.
Šis procesas vadinamas elektromagnetiniu svyravimu. Aprašykime
tai matematiškai:
\begin{align}
  \intertext{potencialui didėjant, srovė mažėja:}
  \frac{Q}{C}
    &= - L\frac{dI}{dt} \label{eq:emsvyravimas_01} \\
  \intertext{žinome, kad:}
  I
    &= \frac{dQ}{dt} \label{eq:emsvyravimas_02} \\
  \intertext{įsistatome \ref{eq:emsvyravimas_02} į
  \ref{eq:emsvyravimas_01}:}
  \frac{Q}{C}
    &= - L \frac{d^{2}Q}{dt^{2}} \\
  \frac{d^{2}Q}{dt^{2}} + \frac{1}{LC}Q &= 0 \\
\end{align}
Gavome paprasto harmoninio svyravimo diferencialinę lygtį, kurios
sprendinys yra:
\begin{equation*}
  Q = Q_{0} \cos \left( \omega t + \varphi \right) \\
\end{equation*}
čia:
\begin{description}
  \item[$\omega$] $= \sqrt{\frac{1}{LC}}$.
\end{description}

TODO: Parodyti, kaip gaunamas periodas.

Raskime srovės išraišką ritėje:
\begin{align*}
  I
    &= \frac{dQ}{dt} \\
    &= - \omega Q_{0} \sin \left( \omega t + \varphi \right) \\
\end{align*}
taip pat gavome, kad aplitudinis srovės stipris yra:
\begin{align*}
  I_{m}
    &= \omega Q_{0} \\
    &= \frac{Q_{0}}{\sqrt{LC}} \\
\end{align*}

TODO: Parodyti elektrinio ir magnetinio laukų kitimus.
